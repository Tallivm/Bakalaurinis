% ------------------------------------------------------------------------------
%                                   PREAMBLE                                  
% ------------------------------------------------------------------------------
\documentclass[a4paper, 12pt]{article}


% ---------------------------- PAGRINDINIAI PAKETAI ----------------------------

\usepackage[utf8]{inputenc}            % naudojama, kai .tex failas UTF-8 koduotės
\usepackage[L7x]{fontenc}              % nurodoma lietuviško teksto koduotė Latin-7
\usepackage[english, lithuanian]{babel}         % nurodoma, kad dokumentas yra lietuviškas

\usepackage{lmodern}                   % dokumente naudojamas šriftas Latin Modern
\usepackage{microtype}                 % optimizuojami atstumai tarp raidžių žodyje
\usepackage{indentfirst}               % atitraukiama pirmoji naujo skyriaus eilutė
\usepackage{icomma}                    % po kablelio skaičiaus viduryje nebus tarpo

\usepackage{amsmath, amssymb, amsthm}  % matematiniai simboliai ir konstrukcijos
\usepackage{mathtools}                 % standartinio AMS paketo amsmath išplėtimas
\usepackage{graphicx}                  % grafinių failų įterpimas ir kiti nustatymai
\usepackage{booktabs}                  % reikalingas tvarkingoms lentelėms sudaryti
\usepackage{caption}                   % paveiksliukų ir lentelių užrašų formavimas
\usepackage{geometry}                  % paraščių ir kitų lapo parametrų nustatymai
\usepackage{hyperref}                  % interaktyvioms nuorodoms dokumente sukurti

\usepackage{float}                     % for float enviroment positions
\usepackage{enumitem}                  % for lists customization

% ---------------------------- DOKUMENTO NUSTATYMAI ----------------------------

\geometry{left = 3cm, top = 2cm, right = 1.5cm, bottom = 2cm} % paketo geometry parametrų nustatymai
\linespread{1.5}                       % tarpas tarp eilučių

\captionsetup{                         % paketo caption parametrų nustatymai (paveiksliuku, lenteliu numeravimas)
    format = hang,                  
    labelfont = bf,             % žodžio pav, lentelė, ... teksto stylius
    textfont = footnotesize,    % caption teksto dydis
    justification = centering, 
    % skip = 0pt,               % atstumas iki paveiksliuko, lentelės, ...
    tablename = lentelė,        % caption pavadinimas prie table
    figurename = pav,           % caption pavadinimas prie figure
    labelsep = period           % atstumas tarp label (pav, ...) iki caption teksto
}

\hypersetup{                           % paketo hyperref parametrų nustatymai (linkai, citavimas)
    unicode = true,
    linktoc = all,
    linktocpage = false, 
    colorlinks = true,       % linkai yra spalvoti
    linkcolor = black,       % \ref color
    filecolor = Maroon, % \href{run:}{} color
    urlcolor = blue,         % url color
    citecolor = blue,        % \cite color          
    breaklinks = true 
    % pdfauthor = {Autorius},  % pdf dokumento meta-duomenys (autorius)
    % pdftitle = {Pavadinimas} % pdf dokumento meta-duomenyse (pavadinimas)
}

\graphicspath{{figures/}}    % paveikliuku folderis (kad būtų galima nurodyti tik pavadinimą)


% ----------------------- PAPILDOMI PAKETAI IR NUSTATYMAI -----------------------


\usepackage{pdfpages}		% pdf puslapių importavimui
\usepackage{titlesec}                  % leidžia keisti skyriaus pavadinimo stilių
\titlelabel{\thetitle.\quad}           % dedamas taškas po skyriaus numeriu tekste
\let\tocnumdot\numberline              % uždeda tašką po skyriaus numeriu turinyje
\def\numberline#1{\tocnumdot{#1.}} 

\usepackage{tocloft}                                  % format and control table of contents
\renewcommand{\cftsecleader}{\cftdotfill{\cftdotsep}} % for dots in sections toc 

\usepackage{bbm}                      % matematinio teksto ir simbolių formatavimas
\usepackage{xfrac}                    % display `small', nice fractions (alternative nicefrac)
\usepackage{siunitx}                  % SI vienetų paketas

% \setlength{\parindent}{0pt}            % no paragraph indention
% \setlength{\parskip}{1em}              % add spave between paragraphs

\newcommand{\subsubsubsection}[1]{\paragraph{#1}\mbox{}\\}  % sukuriamas \subsubsubsection, kuris numeruojamas
% \setcounter{secnumdepth}{4}
% \setcounter{tocdepth}{4}

%\usepackage[dvipsnames]{xcolor}           % papildomos spalvos
%\definecolor{back}{RGB}{245, 245, 245}    % nauja spalva (pilka, tinkanti kodo fonui)

\usepackage{algorithm}              % for writing algorithms
\usepackage[noend]{algpseudocode}   % for writing algorithms

\usepackage{verbatim}               % programinio kodo ir komentaru įterpimas
\usepackage{fancyvrb}               % programinio kodo įterpimas ir formatavimas
\usepackage{fvextra}                % programinio kodo formatavimas
\usepackage{listings}               % programinio kodo įterpimas ir formatavimas (galima naudoti caption)
\renewcommand{\lstlistingname}{išvestis} % lstlisting caption pavadinimas
\lstset
{ %Formatting for code in appendix (paprastas)
    basicstyle = \linespread{1}\tt\small,
    breakatwhitespace = false,
    breaklines = true,
    columns = fullflexible,
    showspaces = false,
    showstringspaces = false,
    showtabs = false,
    tabsize = 4,
    % aboveskip = 5pt,
    % belowskip = 5pt,
    inputencoding=utf8,
    escapeinside = {(*@}{@*)}
}

\usepackage{tabu, multirow} % lenteleje storesnes linijas padaryti, kelias eilutes apjungti
\usepackage{diagbox}

% Spalvoti TODO komentarai
\usepackage[colorinlistoftodos]{todonotes}
\newcommand{\TD}[1]{\todo[inline, color=red!40]{#1}}

% ------------------------- NELABAI REIKALINGI PAKETAI -------------------------
% --------------------------- BET GALI BŪTI NAUDINGI ---------------------------

\usepackage{multicol}                  % leidžia rašyti tekstą į kelis stulpelius
\usepackage{lipsum}                    % for dummy text only
\usepackage{titling}                   % for beutiful title
\usepackage{pifont}                    % special symbols 
\usepackage{physics}                   % for typesetting equations for physics
% \usepackage{subcaption}                % subcaption in figures (not working with subfig)
\usepackage{subfig}                    % for subfigures in figure 
\usepackage{wrapfig}                   % wrap text around figures 


\setlist[enumerate]{ % list formatting
    itemsep = 0.1pt,
    topsep = 5pt
}
\setlist[itemize]{ % list formatting
    itemsep = 0.1pt,
    topsep = 5pt
}

\lstdefinestyle{fancycode}{                        % aplinkos lstlisting nustatymai graziam kodo iterpimui:
                                                   % KODE REIKES NURODYTI [style=fancycode]
    backgroundcolor = \color{back},           % colour for the background
    commentstyle = \it\color{magenta},        % style of comments
    keywordstyle = \color{blue},              % style of keywords
    numberstyle = \tiny\color{black},         % style used for line-numbers
    stringstyle = \color{ForestGreen},        % style of strings
    basicstyle = \linespread{1}\footnotesize, % font size/family/etc. for source 
    breakatwhitespace = false,                % sets if automatic breaks should only happen at whitespaces
    columns = fullflexible,                   % for optimal spacing between characters
    keepspaces = true,                        % keep spaces in the code, useful for indetation
    breaklines = true,                        % automatic line-breaking       
    %captionpos = b,                          % position of caption (t/b)
    numbers = left,                           % position of line numbers (left/right/none)
    numbersep = 12pt,                         % distance of line-numbers from the code
    showspaces = false,                       % emphasize spaces in code 
    showstringspaces = false,                 % emphasize spaces in strings 
    showtabs = false,                         % emphasize tabulators in code
    tabsize = 2,                              % default tabsize
    frame = trBL,                             % showing frame outside code 
%                                             %(none/leftline/topline/bottomline/lines/single/shadowbox/trbl)
    escapeinside = {(*@}{@*)},                % specify characters to escape from source code to LATEX
    rulecolor = \color{gray},                 % Specify the colour of the frame-box
    frameround = fttt,                        % from top-right (t - round)
    %framerule = 2mm,                          % margin of frame
    inputencoding=utf8,
    aboveskip = 20pt,                         % space above 
    belowskip = 15pt                          % space below
}

\newcommand{\file}[1]{\href{run:#1}{#1}}   % komanda failo pavadinimui iterpti, kad ji galima butu atidaryti is pdf
\renewcommand{\eqref}[1]{\hyperref[#1]{\textbf{\footnotesize \ref{#1}}}} % kad eqref numerius boldintu

% \definecolor{black1}{RGB}{95, 95, 95}
% \renewcommand{\_}{{\color{black1} \underline{\hspace{0.35em}}}} % apatinis pabraukimas

% \usepackage{slantsc} % combine two font styles
% \newcommand{\textscit}[1]{\textsl{\textsc{#1}}} % small caps italic