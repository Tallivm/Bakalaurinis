% ŠIO FAILO NEREIKIA KOMPILIUOTI (MES ERRORĄ). KOMPILIUOK "referatas.tex"

\section{METODAI}

\subsection{Duomenų apie molekules surinkimas}

\subsection{Lingadų įvedimas į batymą}

\TD{Naudojama programa - LeDock} 

\subsection{Neuroninio tinklo architektūra}

\subsection{Kompleksų kodavimas ir duomenų padavimas neuroniniam tinklui}



%http://tug.ctan.org/info/symbols/comprehensive/symbols-a4.pdf

%\begin{figure}[H]
%\centering 
%\includegraphics[width=0.55 \linewidth]{failo pavadinimas, galima be plėtinio}
%\caption{Užrašas po paveiksliuko.}
%\label{fig:bla}
%\end{figure}

%\begin{figure}[H]
%\centering 
%\subfloat[Užrašas po pirmo pav\label{fig:abba}]{\includegraphics[width=0.45 \linewidth]{pirmo failo pavadinimas}}
%\subfloat[Užrašas po antro pav\label{fig:baobab}]{\includegraphics[width=0.45 \linewidth]{antro failo pavadinimas}}
%\caption{Bendras užrašas}
%\label{fig:blabla}
%\end{figure}

% \ref{fig:} pav: \href{https://upload.wikimedia.org/wikipedia/commons/2/20/Illu_blood_cell_lineage.jpg}{Eritropoezės schema}