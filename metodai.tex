% !TEX root = referatas.tex

% Sutrumpinimai:
% ReLU - Rectified Linear Unit
% EEM - Electronegativity Equalization Method

\section{METODAI IR MODELIAVIMAI}

\subsection{Duomenų apie molekules surinkimas}

\TD{Pasirinkti ligandai. Pasirinkti 60 proteinai, su kuriais žinomos sąveikos su ligandais. Baltymai turi būti geros kokybės. Trumpai aprašyti baltymų svarbumą. Ligandai turi būti apie 50/50 gerų ir "blogų", geriau be dekojų, bet tikriausiai reikės}

% -----------------------------------------------------------------------------------

\subsection{Lingadų įvedimas į batymą}

Ligandų įvedimas į baltymą atliekamas naudojant \emph{LeDock} programą. Ji buvo pasirinkta todėl, kad parodė aukštus įvertinimus palyginus su kitomis šiam tikslui pritaikytomis programomis.\cite{} \emph{LeDock} naudoja vertinimo funkciją, kuri skaičiuoja komplekso surišimo stiprumą: %http://www.lephar.com/download/LeDock.pdf
\begin{equation}
\Delta G_{bind}=\alpha \sum_{i \in lig}\left(E_{i}^{vdw}+E_{i}^{hb}\right) \Theta\left(E_{c o}-E_{i}^{v d w}-E_{i}^{h b}\right)+\beta(r) \sum_{i \in lig} \sum_{j \in pro} \frac{q_{i} q_{j}}{r_{i j}}+\gamma E_{lig}^{\text {strain}}
\end{equation}
kur $E^{vdw}$ ir $E^{hb}$ yra atitinkamai Van der Valso ir vandenilinių ryšių energijų sumos, $\Theta$ yra \emph{Heaviside} funkcija, $E_{co}$ yra 'cutoff' \TD{?} energijos suma, $q$ yra dalinis atomo krūvis, $r$ - atstumas tarp atomų poros, $\alpha$, $\beta$ ir $\gamma$ yra empiriniai koeficientai.
\TD{jei jis skaičiuoja naudojant dalinius krūvius, kodėl paskui juos reikia vėl skaičiuot?? Atrodo aš baltymui skaičiuoju prieš dockingą krūvius, o ligandams ne? Aš juos konvertuoju iš sdf į mol2, ar nurodžiu kad dar paskaičiuotų krūvius?}
\TD{Naudojama programa - LeDock. Jai reikalinga baltymo stukrūra, ligando struktūra ir aktyvaus centro koordinates. Akt.c.: PDBsum. 30x30x30 A. Koordinačių konvertavimas LeDockui.} 

% -----------------------------------------------------------------------------------

\subsection{Kompleksų kodavimas ir duomenų padavimas neuroniniam tinklui}

Įvedami į neuronini tinklą duomenys yra vektorių rinkinio pavidalo. Kiekvienas vektorius turi informaciją apie ligando atomą, artimiausius du ligando atomus ir du baltymo atomus. Yra aprašomi atomų tipai (pvz. N, C), daliniai krūviai (slankiojančio kablelio skaičiais), atstumai tarp pirmojo atomo ir kitų atomų (slankiojančio kablelio neneigiamas skaičius; pirmasis atomas turės 0), bei aminorūgšties tipai, kuriai priklauso baltymo atomai (pvz. Gln). Atomų daliniai krūviai atvaizduoja elektronų pasiskirstymą molekulėje, kas nulemia molekulės tam tikras chemines savybes.
Kiekvienas kompleksas po ligando įvedimo į baltymą su \emph{LeDock} pagalba turi informaciją apie baltymo ir ligando atomų koordinates trimatėje erdvėje, bei atomų tipus, tačiau neturi informacijos apie atomų dalinius krūvius. Jie yra skaičiuojami naudojant \emph{OpenBabel} įrankį ir pasirenkant EEM algoritmą. % https://www.ncbi.nlm.nih.gov/pmc/articles/PMC3716427/ https://www.ncbi.nlm.nih.gov/pmc/articles/PMC4667495/
Jis buvo pasirinktas todėl, kad pagal \cite{} ir \cite{} straipsnius, EEM algoritmas skaičiuoja dalinius krūvius ypač aukštu tikslumu.

EEM skaičiuoja elektronegatyvumą kiekvienam atomui molekulėje pagal tokią formulę: %https://www.pamoc.it/kw_eem.html

\begin{equation}
\chi_{i}=\left(\chi_{i}^{\circ}+\Delta \chi_{i}\right)+2\left(\eta_{i}^{\circ}+\Delta \eta_{i}\right) q_{i}+k \sum_{i \neq j}^{N} \frac{q_{j}}{r_{i, j}}
\end{equation}
čia $\chi_{i}^{\circ}$ yra izoliuoto atomo $i$ elektronegatyvumas, $\Delta \chi_{i}$ yra jo patikslinimas turint omenyje atomo buvimą molekulėje; $\eta_{i}^{\circ}$ yra atomo kietumas; $q_{i}$ yra atomo krūvis; $k \sum_{i \neq j}^{N} \frac{q_{j}}{r_{i, j}}$ parodo elektrostatinius santykius tarp šio atomo ir kiekvieno kito atomo $j$ molekulėje.

Tuomet bendras molekulės krūvis yra: 

\begin{equation}
Q=\sum_{i=1}^{N} q_{i}
\end{equation}



%ir taikant Gasteigerio algoritmą. Skaičiuojant $Q_i$ dalinį krūvį $i$ atomui, yra tokia formulė\cite{gasteiger_new_1978}:

%\begin{equation}
%Q_{i}=\sum_{a}\left\{\left[\sum_{j} \frac{1}{D_{i}}\left(x_{j}-x_{i}\right)+\sum_{k} \frac{1}{D_{k}}\left(x_{k}-x_{i}\right)\right]\left(\frac{1}{f}\right)^{a-1}\right\}
%\end{equation}
%kur $j$ ir $k$ yra šalia esantys atomai, kurie turi didesnį ar mažesnį elektronegatyvumą; $x_i$, $x_j$ ir $x_k$ yra atitinkamų atomų eletronegatyvumai; $D_i$ yra maksimalus elektronegatyvumo skirtumas, $a$ -- ?, $f$ -- ?


\TD{docking\_data\_processor, aprašyti}

% -----------------------------------------------------------------------------------

\subsection{Neuroninio tinklo architektūra}

%Neuroninis tinklas sudarytas iš daugelio neuronų. Kiekvienas neuronas gauna kelias įvestis, kiekviena iš kurių turi savo svorį. Iš pradžių svorio reikšmės yra parenkamos atsitiktiniu būdu, tačiau keičiamos modeliui besimokant. Neurono produkto skaičiavimo formulė atrodo taip:
%\begin{equation}
%z=\sum_{j=1}^{k} x_{j} w_{k}+b w_{0}
%\end{equation}
%kur $w_k$ yra tam tikros įvesties svoris ir $b$ yra nuolydis (angl. \textit{bias}). Nuolydis yra papildoma įvestis, kurios reikšmė yra fiksuota. Jis yra svarbus modelio lankstumui.

Pradinis neuroninio tinklo kodas yra paimtas iš \cite{pereira_boosting_2016}. Jis yra parašytas \emph{Python} 2.7 programavimo kalba. Šiam darbui jis buvo nežymiai pakeistas, kad atitiktų \emph{Python} 3.7 versijai. Taip pat šio kodo duomenų apdorojimo dalis yra pritaikyta darbui su \emph{Autodock Vina} programos failais, todėl buvo perrašyta tam, kad tiktų darbui su \emph{LeDock} programa. Šis neuroninis tinklas yra pavadintas \emph{DeepVS}. Jis yra konvoliucinis neuroninis tinklas ir operuoja su vektoriais. Tam jis naudoja tiesinį transformavimą, \emph{ReLU}  aktyvacijos funkciją ir \emph{dropout} reguliarizacijos metodą. \emph{Dropout} metodas yra atsitiktinių reikšmių išmetimas (atsitiktinių neuronų išjungimas), kas gerai veikia prieš modelio persimokymą. \emph{DeepVS} naudoja \emph{dropout} metodą su 0,5 tikimybe kiekvienai iš reikšmių būti išmestai. Gale yra taikoma \emph{softmax} funkcija, po kurios gaunamas galutinis baltymo--ligando komplekso įvertinimas.

%http://tug.ctan.org/info/symbols/comprehensive/symbols-a4.pdf

%\begin{figure}[H]
%\centering 
%\includegraphics[width=0.55 \linewidth]{failo pavadinimas, galima be plėtinio}
%\caption{Užrašas po paveiksliuko.}
%\label{fig:bla}
%\end{figure}

%\begin{figure}[H]
%\centering 
%\subfloat[Užrašas po pirmo pav\label{fig:abba}]{\includegraphics[width=0.45 \linewidth]{pirmo failo pavadinimas}}
%\subfloat[Užrašas po antro pav\label{fig:baobab}]{\includegraphics[width=0.45 \linewidth]{antro failo pavadinimas}}
%\caption{Bendras užrašas}
%\label{fig:blabla}
%\end{figure}

% \ref{fig:} pav: \href{https://upload.wikimedia.org/wikipedia/commons/2/20/Illu_blood_cell_lineage.jpg}{Eritropoezės schema}